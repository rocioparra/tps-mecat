% ! Tex root = mecat-sensorfusion-main

\usepackage[utf8]{inputenc}
% caracteres utf8 (tildes, enie) sin tener que usar comandos



\usepackage[T1]{fontenc}

\usepackage[spanish, es-tabla, es-nodecimaldot]{babel} 
% texto automatico en espaniol
% "tabla" en vez de "cuadro"
% no reemplaza puntos decimales por comas

%% NO AGREGAR PAQUETES ANTES DE ESTO, ES IMPORTANTE QUE BABEL ESTE PRIMERO

%%%%%%%%%%%%%%%%%%%%%%%%%%%%%%%%%
%% PAQUETES EXTRA %%%%%%%%%%%%%%%
%%%%%%%%%%%%%%%%%%%%%%%%%%%%%%%%%

\usepackage{hyperref}					% Hyperlinks on pdf (must be called before Geometry)

\usepackage[a4paper, total={6in, 9in}, footskip=25px]{geometry} 

\usepackage{sansmathfonts}				% Sans Serif equations

\renewcommand*\familydefault{\sfdefault} 	% Sans Serif as default font



\usepackage{subfiles}
\usepackage{xr} % permite referencias a labels de archivos externos con \externaldocument{filename.tex}

\usepackage{amsmath} % PAQUETES DE MATEMATICA
\usepackage{amsfonts}
\usepackage{amssymb}


\usepackage{booktabs} % tablas lindas

\usepackage{units} % permite usar nicefrac
\usepackage{siunitx}
\usepackage{graphicx} % importar imagenes
\usepackage{float} % posicion H para floats
\usepackage[colorinlistoftodos]{todonotes}


\setlength{\parindent}{10pt}			%cuanta sangria al principio de un parrafo
\usepackage{indentfirst}				%pone sangria al primer parrafo de una seccion


% Header style
\usepackage{fancyhdr}
\setlength{\headheight}{15.2pt}
\pagestyle{fancy}
\lhead{31.99 Mecatr\'onica Aplicada}
\chead{TP Sensor Fusion}
\rhead{Roc\'io Parra}
\cfoot{\thepage}


%\usepackage{dblfnote}
%\DFNalwaysdouble 


\hypersetup{
	colorlinks=true,
	linkcolor=blue,
	filecolor=magenta,      
	urlcolor=blue,
	citecolor=blue,    
}

%Para los graficos con multiples imagenes en el mismo float
\usepackage{caption}
\usepackage{subcaption}

\usepackage{wrapfig} % figuras wrappeadas por texto
\usepackage{verbatim} % comment and verbatim environments

\usepackage{listings}
\usepackage{xcolor}

\definecolor{codegreen}{rgb}{0,0.6,0}
\definecolor{codegray}{rgb}{0.5,0.5,0.5}
\definecolor{codepurple}{rgb}{0.58,0,0.82}
\definecolor{backcolour}{rgb}{0.95,0.95,0.92}


\lstdefinestyle{mystyle}{
	backgroundcolor=\color{white},   
	commentstyle=\color{codegreen},
	keywordstyle=\color{magenta},
	numberstyle=\tiny\color{codegray},
	stringstyle=\color{codepurple},
	basicstyle=\ttfamily\footnotesize,
	breakatwhitespace=false,         
	breaklines=true,                 
	captionpos=b,                    
	keepspaces=true,                 
	numbers=left,                    
	numbersep=5pt,                  
	showspaces=false,                
	showstringspaces=false,
	showtabs=false,                  
	tabsize=2
}

\lstset{style=mystyle}

 